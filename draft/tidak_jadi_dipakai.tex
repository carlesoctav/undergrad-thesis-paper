\subsection{Regresi Kernel Sebagai \f{Attention} non-parametrik}
\label{sec:regresi-kernel}

Salah satu pengunaan mekanisme \f{attention} terdapat pada regresi kernel yang merupakan model statistik non-parametrik. \f{Attention} berubah menjadi regresi kernel dengan memilih fungsi atensi $f_{attn}(\mathbf{q}, \mathbf{k})$ menjadi fungsi non-parametrik $\mathcal{K}(\mathbf{q}, \mathbf{k})$.

Pada model non-parametrik, model yang dibangun tidak memiliki parameter yang harus diestimasi atau dipelajari, melainkan model non-parametrik menggunakan seluruh atau sebagian dari \f{datasets} latih $\mathcal{D} = \{(\mathbf{x}_1, y_1), (\mathbf{x}_2, y_2), \dots, (\mathbf{x}_n, y_n)\}$ dalam memberikan prediksi $y_*$ pada sebuah data uji $\mathbf{x}_*$. \equ~\ref{equ:regresi-kernel-start} hingga \equ~\ref{equ:regresi-kernel-end} menunjukkan bagaimana model regresi kernel melakukan prediksi pada sebuah data uji $\mathbf{x}_*$.

\begin{align}
    \label{equ:regresi-kernel-start}
    y_{*} = f(\mathbf{x}_{*}) &= \sum_{i=1}^{n} \alpha_{i}(\mathbf{x}_{*},\mathbf{x}_i) y_i, \\
    \label{equ:regresi-kernel-end}
    \text{dengan } \alpha_{i}(\mathbf{x}_{*},\mathbf{x}_i) &= \frac{\mathcal{K}(\mathbf{x}_{*},\mathbf{x}_i)}{\sum_{j=1}^{n} \mathcal{K}(\mathbf{x}_{*},\mathbf{x}_j)} \in [0, 1].
\end{align}

$\mathcal{K}(\mathbf{x}_{*},\mathbf{x}_i)$ adalah fungsi kernel (non-parametrik) yang menghitung keserupaan antara data uji $\mathbf{x}_{*}$ dengan data latih $\mathbf{x}_i$. \equ~\ref{equ:regresi-kernel-end} merupakan bentuk khusus dari mekansime \f{soft attention} \equ~\ref{equ:soft-attention-start}, dengan kueri $\mathbf{q}$ adalah data uji $\mathbf{x}_{*}$, kunci $\mathbf{k}_i$ adalah data latih $\mathbf{x}_i$, nilai $\mathbf{v}_i$ adalah $y_i$. \equ~\ref{equ:regresi-kernel-gaussian-start} menujukkan contoh kernel regresi dengan pemilihan kernel Gaussian $\mathcal{K}(\mathbf{x}_{*},\mathbf{x}_i) = \exp(-\frac{||\mathbf{x}_{*} - \mathbf{x}_i||^2 \beta}{2})$.

\begin{align}
    \label{equ:regresi-kernel-gaussian-start}
    y_{*} = f(\mathbf{x}_{*}) &= \sum_{i=1}^{n} \alpha_{i}(\mathbf{x}_{*},\mathbf{x}_i) y_i	\\
    &= \sum_{i=1}^{n} \frac{\exp\left(-\frac{||\mathbf{x}_{*}-\mathbf{x}_i||^2 \beta^2}{2}\right)}{\sum_{j=1}^{n} \exp\left(-\frac{||\mathbf{x}_{*}-\mathbf{x}_j||^2 \beta^2}{2}\right)} y_i
\end{align}