%-----------------------------------------------------------------------------%
\addappendix{Kode Simulasi}
\chapter*{Lampiran 1: Kode Simulasi}
\begin{enumerate}
    \item Repositori kode: \url{https://github.com/carlesoctav/beir-skripsi} 
    \item Repositori model dan data: \url{https://huggingface.co/carles-undergrad-thesis}
    \item Repositori data (\f{raw}): \url{https://drive.google.com/drive/folders/1l_fbqJSn2AR8f-g1QnANl5czm2aLO0rO}
\end{enumerate}

\lstinputlisting[language=Python, caption=Kode untuk mengevaluasi $\text{BERT}_{\text{DOT}}$, label=code:python]{assets/codes/beir-skripsi/experiments/hardnegs_bm25/eval_miracl_hardnegs.py}

\newpage

\lstinputlisting[language=Python, caption=Kode untuk mengevaluasi $\text{BERT}_{\text{CAT}}$, label=code:python]{assets/codes/beir-skripsi/experiments/bertcat_trained/eval_mrtydi_bertcat.py}

\newpage

\lstinputlisting[language=Python, caption=Kode untuk mengevaluasi BM25, label=code:python]{assets/codes/beir-skripsi/experiments/bm25/eval_miracl_bm25.py}

\newpage

\lstinputlisting[language=Python, caption=Kode untuk melatih  $\text{IndoBERT}_{\text{CAT}}$, label=code:python]{assets/codes/beir-skripsi/experiments/bertcat_trained/train.py}

\newpage

\lstinputlisting[language=Python, caption=Kode untuk melatih $\text{IndoBERT}_{\text{DOT}}$, label=code:python]{assets/codes/beir-skripsi/experiments/in_batch/train.py} 

\newpage

\lstinputlisting[language=Python, caption=Kode untuk melatih $\text{IndoBERT}_{\text{DOTKD}}$, label=code:python]{assets/codes/beir-skripsi/experiments/knowledge-distill-train/train.py}





%-----------------------------------------------------------------------------%

%-----------------------------------------------------------------------------%
\addappendix{Output dari Simulasi}
\chapter*{Lampiran 2: \f{Output} dari Simulasi}
\label{appendix:output}
Terlampir contoh \f{output} pengevaluasian dari suatu model, Informasi lebih lanjut dapat dilihat di \url{https://github.com/carlesoctav/beir-skripsi}.

\inpdf{assets/pdfs/mmarco-report}