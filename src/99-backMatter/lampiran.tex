%-----------------------------------------------------------------------------%
\addappendix{CHANGELOG}
\chapter*{Lampiran 1: Kode Simulasi}
\begin{enumerate}
    \item Repositori kode: \url{https://github.com/carlesoctav/beir-skripsi} 
    \item Repositori model dan data: \url{https://huggingface.co/carles-undergrad-thesis}
    \item Repositori data (\f{raw}): \url{https://drive.google.com/drive/folders/1l_fbqJSn2AR8f-g1QnANl5czm2aLO0rO}
\end{enumerate}




%-----------------------------------------------------------------------------%

%-----------------------------------------------------------------------------%
\addappendix{Judul Lampiran 2}
\chapter*{Lampiran 2: Judul Lampiran 2}
\label{appendix:sample}
%-----------------------------------------------------------------------------%
Lampiran hadir untuk menampung hal-hal yang dapat menunjang pemahaman terkait tugas akhir, namun akan mengganggu \f{flow} bacaan sekiranya dimasukkan ke dalam bacaan.
Lampiran bisa saja berisi data-data tambahan, analisis tambahan, penjelasan istilah, tahapan-tahapan antara yang bukan menjadi fokus utama, atau pranala menuju halaman luar yang penting.

%-----------------------------------------------------------------------------%
\section*{Subbab dari Lampiran 2}
\label{appendix:sampleSubchap}
%-----------------------------------------------------------------------------%
\todo{Isi subbab ini sesuai keperluan Anda. Anda bisa membuat lebih dari satu judul lampiran, dan tentunya lebih dari satu subbab.}
