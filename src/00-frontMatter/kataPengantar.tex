%-----------------------------------------------------------------------------%
\chapter*{\kataPengantar}
%-----------------------------------------------------------------------------%
\pagestyle{first-pages}

Segala Puji dan Syukur penulis panjatkan kepada Tuhan Yang Maha Esa ats diberikan anugerah dan kesempatan sehingga penulis dapat menyelesaikan skripsi yang berjudul "\judul". Penulisan skripsi ini dilakukan dalam rangka memenuhi salah satu syarat kelulusan dan gelar Sarjana Matematika pada Fakultas Matematika dan Ilmu Pengetahuan Alam, Universitas Indonesia. Penyusunan skripsi ini didasari atas semangat, usaha dan doa kepada-Nya. Dalam proses Penyusunan skripsi, penulis juga tidak lepas dari bantuan orang sekitar, baik berupa dukungan, bimbingan, dan doa yang telah diberikan. Penulis juga mengucapkan terima kasih sebesar-besarnya kepada:

\begin{enumerate}
	\item \pembimbingSatu, selaku dosen pembimbing yang banyak memberikan arahan, saran dan bantuan dalam menyelesaikan skripsi ini.
	\item Orang tua penulis yang selalu memberikan doa, kasih sayang, serta dukungan berupa moril maupun materiil yang tak terhingga. 
	\item Bapak dan Ibu dosen dan staf pengajar Matematika Universitas Indonesia yang telah mengajarkan penulis berbagai macam ilmu.
	\item Anthony, Antonius yang selalu menemani, mendukung dan memberikan semangat selama penyusunan skripsi.
	\item Teman-teman penulis selama perkuliahan, yaitu Nicholas, Bravy, Owen, Gladys.
	\item Komunitas \f{Machine learning} yang berkontribusi menyediakan sumber daya secara gratis dan terbuka sehingga membantu penulis dalam penelitian ini, baik dari dasar teori hingga tahap implementasi.
	\item Pihak-pihak yang sudah membantu penulis dalam melakukan penelitian ini, menyusun skripsi, dna mendukung dalam dunia perkuliahan baik secara langsung maupun tidak langsung.
\end{enumerate}

Akhir kata, penulis memohon maaf atas kekurangan dalam pengerjaan dan penulisan skripsi ini. Penulis menyadari bahwa skripsi ini masih belum sempurna. Oleh karena itu, kritik dan saran yang bersifat membangun sangat penulis harapkan. Semoga peneletian ini dapat bermanfaat bagi pengembangan ilmu pengetahuan kedepannya dan bagi pihak-pihak terkait.
 
% Untuk input gambar tanda tangan, silahkan sesuaikan xshift, yshift, dan width dengan gambar tanda tangan Anda
%\begin{tikzpicture}[remember picture,overlay,shift={(current page.north east)}]
%\node[anchor=north east,xshift=-3cm,yshift=-6.2cm]{\includegraphics[width=3cm]{assets/pics/tanda_tangan_wikipedia.png}};
%\end{tikzpicture}

\vspace*{0.1cm}
\begin{flushright}
Depok, \tanggalSiapSidang\\[0.1cm]
\ifx\blank\npmDua
	\vspace*{1.5cm}
	\penulisSatu
\else
	Tim Penulis
\fi

\end{flushright}
