%
% Halaman Abstrak
%
% @author  Andreas Febrian
% @version 2.1.2
% @edit by Ichlasul Affan
%

\chapter*{Abstrak}
\singlespacing

\vspace*{0.2cm}

\noindent \begin{tabular}{l l p{10cm}}
	\ifx\blank\npmDua
		Nama&: & \penulisSatu \\
		Program Studi&: & \programSatu \\
	\else
		Nama Penulis 1 / Program Studi&: & \penulisSatu~/ \programSatu\\
		Nama Penulis 2 / Program Studi&: & \penulisDua~/ \programDua\\
	\fi
	\ifx\blank\npmTiga\else
		Nama Penulis 3 / Program Studi&: & \penulisTiga~/ \programTiga\\
	\fi
	Judul&: & \judul \\
	Pembimbing&: & \pembimbingSatu \\
	\ifx\blank\pembimbingDua
    \else
        \ &\ & \pembimbingDua \\
    \fi
    \ifx\blank\pembimbingTiga
    \else
    	\ &\ & \pembimbingTiga \\
    \fi
\end{tabular} \\

\vspace*{0.5cm}

\noindent Peningkatan jumlah data teks digital membuat manusia membutuhkan mekanisme untuk mengembalikan teks yang efektif dan efisien. Salah satu mekanisme untuk mengembalikan teks adalah dengan pemeringkatan teks. Tujuan dari pemeringkatan teks adalah menghasilkan daftar teks yang terurut berdasarkan relevansinya dalam menanggapi permintaan kueri pengguna. Pada penelitian ini, penulis menggunakan \f{Bidirectional Encoder Representations from Transformers} (BERT) untuk membangun model pemeringkatan teks berbahasa Indonesia. Penggunaan BERT memberikan peningkatan kualitas pemeringkatan teks bila dibandingkan dengan model \f{baseline} BM25. Peningkatan kualitas pemeringkatan teks tersebut dapat dilihat dari nilai metrik \f{recriprocal rank} (RR), \f{recall} (R), dan \f{normalized discounted cumulative gain} (NDCG). 

\vspace*{0.2cm}

\noindent Kata kunci: \\ IndoBERT, representasi teks, sistem temu balik, skoring teks\\

\setstretch{1.4}
\newpage
