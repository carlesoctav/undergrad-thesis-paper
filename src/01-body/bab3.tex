\chapter{\babTiga}
\label{bab:3}

\noindent\todo{
	jabarin sih isinya mau gmna
}

\section{Mekanisme \f{Attention}}
	\subsection{Attention sebagai \f{Dictionary Lookup}}

	Mekanisme \f{Attention} dapat ditinjau sebagai \f{Dictinoary Lookup}, yaitu untuk sebuah vektor kueri $\mathbf{q}$ dan sekumpulan pasangan terurut vektor $\mathcal{KV} = \{(\mathbf{k}_1, \mathbf{v}_2), (\mathbf{k}_2, \mathbf{v}_2), \dots, (\mathbf{k}_n, \mathbf{v}_n)\}$, mekanisme \f{attention} akan mengembalikan vektor nilai $\mathbf{v}_i$ yang memiliki vektor kunci $\mathbf{k}_i$ yang serupa dengan vektor kueri $\mathbf{q}$. \equ~\ref{equ:hard-attention-start} hingga \equ~\ref{equ:hard-attention-end} menunjukkan bagaimana mekanisme \f{attention} dilakukan.

\begin{align}
	\label{equ:hard-attention-start}
	\mathcal{KV} &= \{(\mathbf{k}_1, \mathbf{v}_2), (\mathbf{k}_2, \mathbf{v}_2), \dots, (\mathbf{k}_n, \mathbf{v}_n)\}, \\
	\text{tulis kembali }\mathbf{K}&= \begin{bmatrix}
		\mathbf{k}_1 \\
		\mathbf{k}_2 \\
		\vdots \\
		\mathbf{k}_n
	\end{bmatrix} \in \mathbb{R}^{n \times d_k}, \\
	\mathbf{V} &= \begin{bmatrix}
		\mathbf{v}_1 \\
		\mathbf{v}_2 \\
		\vdots \\
		\mathbf{v}_n
	\end{bmatrix} \in \mathbb{R}^{n \times d_v}, \\
	\text{Attention}(q, \mathbf{K}, \mathbf{V}) &= \mathbf{\alpha}\mathbf{V} \in \mathbb{R}^{d_v},\\
	\mathbf{\alpha} &= [\alpha_{1}, \alpha_{2}, \dots, \alpha_{n}], \\
	\label{equ:hard-attention-end}
	\text{dengan } \alpha_i &= 
	\begin{cases}
	1, & \text{jika } i = \arg\max_{j} f_{attn}(\mathbf{q}, \mathbf{k}_j) \\
	0, & \text{lainnya}
	\end{cases},
	\end{align}
	

	dan $f_{attn}(\mathbf{q}, \mathbf{k})$ adalah fungsi yang menghitung nilai keserupaan antara vektor kueri $\mathbf{q}$ dan vektor kunci $\mathbf{k}$. $\alpha_i$ pada persamaan di atas disebut sebagai bobot atensi dan nilai $f_{attn}(\mathbf{q}, \mathbf{k})$ disebut sebagai nilai atensi. 


	Kasih contoh hard attention.


	Mekansime \f{attention} pada \equ~\ref{equ:hard-attention-start} hingga \equ~\ref{equ:hard-attention-end} disebut sebagai \f{hard attention} karena hanya satu vektor nilai $\mathbf{v}_i$ yang dipilih dari sekumpulan vektor nilai $\mathbf{V}$. Berbeda dengan \f{hard attention} yang tidak terturunkan, akibatnya \f{hard attention} tidak dapat dilatih dengan \f{backpropagation}, \f{soft attention} mengambil seluruh vektor nilai $\mathbf{V}$ dan menghitung bobot $\alpha_i$ untuk setiap vektor nilai $\mathbf{v}_i$ dengan fungsi softmax. Hasil dari \f{soft attention} adalah rata-rata terbobot dari seluruh vektor nilai $\mathbf{V}$. \equ~\ref{equ:soft-attention-start} dan gambarxx menunjukkan bagaimana \f{soft attention} dilakukan.


	\begin{align}
		\label{equ:soft-attention-start}
		\text{Attention}(q, \mathbf{K}, \mathbf{V}) &= \mathbf{\alpha}\mathbf{V} \in \mathbb{R}^{d_v},\\
		\text{dengan } \mathbf{\alpha} &= [\alpha_{1}, \alpha_{2}, \dots, \alpha_{n}], \\
		\text{dan } \alpha_{i}(\mathbf{q},\mathbf{k}_i) &= \text{Softmax}_i(\mathbf{\alpha}) = \frac{\exp(f_{attn}(\mathbf{q}, \mathbf{k}_i))}{\sum_{j=1}^{n} \exp(f_{attn}(\mathbf{q}, \mathbf{k}_j))}, \\
		\sum_{i=1}^{n} \alpha_{i} &= 1, \\
		\label{equ:soft-attention-end}
		0 \leq \alpha_{i} &\leq 1.
	\end{align}

	Dengan rata-rata terbobot dari $\mathbf{V}$, \f{soft attention} dapat dilatih dengan \f{backpropagation} yang merupakan syarat \f{fundamental} yang harus dimiliki oleh sebuah model \f{deep learning}.

	kasih contoh soft attention.

	Pada kasus kumpulan kueri $\mathcal{Q} = \{\mathbf{q}_1, \mathbf{q}_2, \dots, \mathbf{q}_m\}$, Perhitungan atensi untuk setiap triplet $(\mathbf{q}_i, \mathbf{K}, \mathbf{V})$ dapat dihitung secara bersamaan dengan menggunakan operasi matriks. \equ~\ref{equ:soft-attention-start} hingga \equ~\ref{equ:soft-attention-end} yang digunakan untuk kasus 1 kueri dapat ditulis ulang seperti pada \equ~\ref{equ:soft-attention-matrix-start} hingga \equ~\ref{equ:soft-attention-matrix-end}.

	\begin{align}
		\label{equ:soft-attention-matrix-start}
		\text{tulis }\mathbf{Q} &= \begin{bmatrix}
			\mathbf{q}_1 \\
			\mathbf{q}_2 \\
			\vdots \\
			\mathbf{q}_m
		\end{bmatrix} \in \mathbb{R}^{m \times d_k}, \\
		\text{Attention}(\mathbf{Q}, \mathbf{K}, \mathbf{V}) &= \mathbf{A} \mathbf{V} \in \mathbb{R}^{m \times d_v},\\
		\mathbf{A} &= \begin{bmatrix}
			\mathbf{\alpha}_1 \\
			\mathbf{\alpha}_2 \\
			\vdots \\
			\mathbf{\alpha}_m
		\end{bmatrix} = \begin{bmatrix}
			\alpha_{11} & \alpha_{12} & \dots & \alpha_{1n} \\
			\alpha_{21} & \alpha_{22} & \dots & \alpha_{2n} \\
			\vdots & \vdots & \ddots & \vdots \\
			\alpha_{m1} & \alpha_{m2} & \dots & \alpha_{mn} \\
		\end{bmatrix} \in \mathbb{R}^{m \times n}, \\
		\label{equ:soft-attention-matrix-end}
		\alpha_{ij}(\mathbf{q}_i, \mathbf{k}_j) &= \text{Softmax}_j(\mathbf{\alpha}_i) = \frac{\exp(f_{attn}(\mathbf{q}_i, \mathbf{k}_j))}{\sum_{k=1}^{n} \exp(f_{attn}(\mathbf{q}_i, \mathbf{k}_k))},
	\end{align}

	dengan $\alpha_{ij}$ adalah bobot yang menunjukkan bobot atensi antara vektor kueri $\mathbf{q}_i$ dengan vektor kunci $\mathbf{k_j}$. 

	\subsection{Regresi Kernel Sebagai \f{Attention} non-parametrik}
	\label{sec:regresi-kernel}

	Salah satu pengunaan mekanisme \f{attention} terdapat pada regresi kernel, yang merupakan model statistik non-parametrik. \f{Attention} berubah menjadi regresi kernel dengan memilih fungsi keserupaan $f_{attn}(\mathbf{q}, \mathbf{k})$ menjadi fungsi non-parametrik $\mathcal{K}(\mathbf{q}, \mathbf{k})$, dan mengganti fungsi softmax menjadi fungsi normalisasi standar, seperti pada \equ~\ref{equ:regresi-kernel-end}.
	
	Pada model non-parametrik, model yang dibangun tidak memiliki parameter yang harus dicari atau dipelajari, melainkan model non-parametrik menggunakan seluruh atau sebagian dari \f{datasets} latih $\mathcal{D} = \{(\mathbf{x}_1, y_1), (\mathbf{x}_2, y_2), \dots, (\mathbf{x}_n, y_n)\}$ untuk memberikan prediksi $y_*$ untuk sebuah data uji $\mathbf{x}_*$. \equ~\ref{equ:regresi-kernel-start} hingga \equ~\ref{equ:regresi-kernel-end} menunjukkan bagaimana model regresi kernel melakukan prediksi $y_*$ untuk sebuah data uji $\mathbf{x}_*$.

	\begin{align}
		\label{equ:regresi-kernel-start}
		y_{*} = f(\mathbf{x}_{*}) &= \sum_{i=1}^{n} \alpha_{i}(\mathbf{x}_{*},\mathbf{x}_i) y_i, \\
		\label{equ:regresi-kernel-end}
		\text{dengan } \alpha_{i}(\mathbf{x}_{*},\mathbf{x}_i) &= \frac{\mathcal{K}(\mathbf{x}_{*},\mathbf{x}_i)}{\sum_{j=1}^{n} \mathcal{K}(\mathbf{x}_{*},\mathbf{x}_j)} \in [0, 1],
	\end{align}

	dan $\mathcal{K}(\mathbf{x}_{*},\mathbf{x}_i)$ adalah fungsi kernel (non-parametrik) yang menghitung keserupaan antara data uji $\mathbf{x}_{*}$ dengan data latih $\mathbf{x}_i$.\equ~\ref{equ:regresi-kernel-end} merupakan bentuk khusus dari mekansime \f{soft attention} \equ~\ref{equ:soft-attention-start}, dengan kueri $\mathbf{q}$ adalah data uji $\mathbf{x}_{*}$, kunci $\mathbf{k}_i$ adalah data latih $\mathbf{x}_i$, nilai $\mathbf{v}_i$ adalah $y_i$. Gambarxx dan \equ~\ref{equ:regresi-kernel-gaussian-start} menujukkan contoh kernel regresi dengan pemilihan kernel Gaussian $\mathcal{K}(\mathbf{x}_{*},\mathbf{x}_i) = \exp(-\frac{||\mathbf{x}_{*} - \mathbf{x}_i||^2 \beta}{2})$.

	\begin{align}
		\label{equ:regresi-kernel-gaussian-start}
		y_{*} = f(\mathbf{x}_{*}) &= \sum_{i=1}^{n} \alpha_{i}(\mathbf{x}_{*},\mathbf{x}_i) y_i	\\
		&= \sum_{i=1}^{n} \frac{\exp\left(-\frac{||\mathbf{x}_{*}-\mathbf{x}_i||^2 \beta^2}{2}\right)}{\sum_{j=1}^{n} \exp\left(-\frac{||\mathbf{x}_{*}-\mathbf{x}_j||^2 \beta^2}{2}\right)} y_i
	\end{align}

	\subsection{\f{Attention} Parametrik}

	Mekanisme \f{attention} yang dilakukan oleh \cite{transformerori} merupakan mekanisme \f{attention} parametrik. Salah satu alasan penggunaan $f_{attn}$ yang parametrik adalah pemilihan fungsi $f_{attn}$ yang non-parametrik seperti pada \sect~\ref{sec:regresi-kernel} memiliki kelemahan:


	\begin{enumerate}
		\item Relasi antar vektor kueri $\mathbf{q}$ dan vektor kunci $\mathbf{k}$ harus diketahui sebelumnya untuk memilih fungsi $f_{attn}$ yang tepat.
		\item Prediksi $y_*$ memerlukan seluruh data latih $\mathcal{D}$, $O(|\mathcal{D}|)$ komputasi diperlukan untuk melakukan satu prediksi.
	\end{enumerate}

	

	Pada mekanisme \f{attention} parametrik, nilai vektor kueri $\mathbf{q}$ dan $\mathbf{v}$ dibandingkan pada ruang vektor yang akan dipelajari (\f{learned embedding space}) daripada ruang vektor aslinya. Sebagai contoh, untuk suatu kueri $\mathbf{q}\in \mathbb{R}^{d_q}$, dan vektor kunci $\mathbf{k} \in \mathbb{R}^{d_k}$, \f{additive attention} yang diperkenalkan oleh \cite{bahdanau2016neural} menghitung nilai keserupaan antara $\mathbf{q}$ dan $\mathbf{k}$ seperti pada \equ~\ref{equ:additive-attention}

	\begin{align}
	\label{equ:additive-attention}
	f_{attn}(\mathbf{q}, \mathbf{k}) &= (\mathbf{q} \mathbf{W}^q  + \mathbf{k} \mathbf{W}^k)  \mathbf{W}^{\text{out}} \in \mathbb{R}, \\
	\text{dengan } &\mathbf{W}^q \in \mathbb{R}^{d_q \times d_{\text{attn}}}, \mathbf{W}^k \in \mathbb{R}^{d_k \times d_{\text{attn}}}, \mathbf{W}_{\text{out}} \in \mathbb{R}^{d_{\text{attn}} \times 1},
	\end{align}

	Dengan $\mathbf{W}^q$, $\mathbf{W}^k$, dan $\mathbf{W}^{\text{out}}$ adalah matriks parameter bobot yang akan dicari atau dipelajari selama proses pelatihan. Contoh parametrik \f{attention} yang lebih sederhana adalah \f{dot-product attention}. Fungsi $f_{attn}$ yang digunakan adalah perkalian titik antara $\mathbf{q}$ dan $\mathbf{k}$ di ruang vektor yang dipelajari (\f{learned embedding space}). \equ~\ref{equ:dot-product-attention} menunjukkan bagaimana \f{dot-product attention} dihitung.

	\begin{align}
		\label{equ:dot-product-attention}
		f_{attn}(\mathbf{q}, \mathbf{k}) = (\mathbf{q} \mathbf{W}^q) (\mathbf{k} \mathbf{W}^k)^{\top}\\
		\text{dengan } \mathbf{W}^q \in \mathbb{R}^{d_q \times d_{\text{attn}}}, \mathbf{W}^k \in \mathbb{R}^{d_k \times d_{\text{attn}}}.
	\end{align}

\section{Transformer}

	\f{Transformers} merupakan Arsitektur \f{deep learning} yang pertama kali diperkenalkan oleh \cite{transformerori}. Awalnya Transformers merupakan model \f{sequance to sequance} yang diperuntukkan untuk permasalahan mesin translasi neural (\f{neural machine translation}). Namun, sekarang \f{transformer} juga digunakan untuk permasalahan pemrosesan bahasa alami lainnya. model-model yang berarsitektur \f{transformer} menjadi model \f{state-of-the-art} untuk permasalahan pemrosesan bahasa alami lainnya, seperti \f{question answering}, \f{sentiment analysis}, dan \f{named entity recognition}.
 
	Berbeda dengan arsitektur mesin translasi terdahulu, transformer tidak mengunakan \f{recurrent neural network} (RNN) atau \f{convolutional neural network} (CNN), melainkan transformer adalah model \f{feed foward network} yang dapat memproses seluruh \f{input} pada barisan secara paralel. Untuk menggantikan kemampuan RNN dalam mempelajari ketergantungan antar \f{input} yang berurutan dan kemampuan CNN dalam mempelajari fitur lokal, transformer bergantung pada mekanisme \f{attention}.

	Terdapat tiga jenis \f{attention} yang digunakan dalam model \f{transformer} \citep{transformerori}:
	\begin{enumerate}
		\item \f{Encoder self-attention}: menggunakan barisan \f{input} yang berupa barisan token atau kata sebagai masukan untuk menghasilkan barisan representasi kontekstual, berupa vektor, dari \f{input}. Setiap representasi token tersebut memiliki ketergantungan dengan token lainnya dari barisan \f{input}.
		\item \f{Decoder self-attention}: menggunakan barisan \f{target} yang berupa kalimat terjemahan parsial, barisan token, sebagai masukan untuk menghasilkan barisan representasi kontekstual (vektor) dari \f{target}. Setiap representasi token tersebut memiliki ketergantungan dengan token sebelumnya dalam urutan masukan.
		\item \f{Decoder-encoder attention}: menggunakan barisan representasi kontekstual dari \f{input}, dan barisan representasi kontekstual dari \f{target} untuk menghasilkan token berikutnya yang merupakan hasil prediksi dari model. Barisan \f{target} yang digabung dengan token hasil prediksi tersebut akan menjadi barisan \f{target} untuk prediksi selanjutnya.
	\end{enumerate}

	\subsection{\f{Token Embedding}}

	Perlu diingat kembali bahwa \f{input} dari \f{Attention} (dan tentunya \f{transformer}) adalah barisan vektor. Jika \f{Attention} ingin dapat digunakan pada permasalahan bahasa, barisan kata atau subkata (selanjutnya disebut token) harus terlebih dahulu diubah menjadi barisan vektor.

	Representasi vektor dari token yang paling sederhana adalah dengan \f{one-hot encoding}. Andaikan $\mathcal{T} = \{t_1, t_2, \dots, t_{|\mathcal{T}|}\}$ adalah semua kemungkinan token yang mungkin muncul dalam permasalahan bahasa yang ingin diselesaikan. Untuk sembarang barisan token $\mathbf{t} = [t_{i_1}, t_{i_2}, \dots, t_{i_L}]$, representasi vektor dari token $t_{i_j}$ adalah vektor $\mathbf{oh}_{i_j} = [0, \dots, 0, 1, 0, \dots, 0] \in \mathbb{R}^{|\mathcal{T}|}$, dengan nilai 1 pada indeks ke $i_j$ dan nilai 0 pada indeks lainnya. \f{One-hot encoding} Tentunya memiliki kelemahan:

	\begin{enumerate}
		\item Vektor yang dihasilkan adalah \f{sparse vector}, dan ukuran vektor yang dihasilkan cukup besar, yaitu $|\mathcal{T}|$.
		\item Representasi token yang buruk. Operasi vektor yang dilakukan pada \f{one-hot encoding} tidaklah bermakna. Misalnya, Jarak antar token akan selalu sama pada \f{one-hot encoding}, yaitu $\sqrt{2}$.
	\end{enumerate}

	Vektor yang padat (\f{dense}) dan memiliki representasi token yang baik adalah vektor yang dinginkan. Representasi vektor yang baik diharapkan dapat dipelajari selama proses pelatihan model. Misalkan $\mathbf{E}_{\mathcal{T}} \in \mathbb{R}^{|\mathcal{T}| \times d_{\text{token}}}$ adalah matriks parameter yang merupakan representasi vektor padat dari seluruh token ada. \equ~\ref{equ:token-embedding-start} hingga \equ~\ref{equ:token-embedding-end} menunjukkan bagaimana representasi vektor dari barisan suatu token $\mathbf{t}$ dihitung. 

	\begin{align}
		\label{equ:token-embedding-start}
		\mathbf{t} &= [t_{i_1}, t_{i_2}, \dots, t_{i_L}], \\
		e_{i_j} &= \mathbf{oh}_{i_j} \mathbf{E}_{\mathcal{T}} \in \mathbb{R}^{d_{\text{token}}}, \\
		\label{equ:token-embedding-end}
		\text{Embed}(\mathbf{t}) &= \mathbf{E}_{\mathbf{t}} = \begin{bmatrix}
			e_{i_1} \\
			e_{i_2} \\
			\vdots \\
			e_{i_L}
		\end{bmatrix} \in \mathbb{R}^{L \times d_{\text{token}}}.
	\end{align}

	\subsection{\f{Scaled Dot-Product Attention}}
	\f{Scaled dot-product attention} adalah mekanisme \f{Attention} parametrik yang digunakan dalam \f{transformers}. \f{Scaled dot-product attention} menghitung keserupaan antara vektor kueri $\mathbf{q}$ dan vektor kunci $\mathbf{k}$ pada ruang vektor yang dipelajari (\f{learned embedding space}) dengan fungsi keserupaan $f_{attn}(\mathbf{q}, \mathbf{k})$ adalah perkalian titik antara $\mathbf{q}$ dan $\mathbf{k}$ yang kemudian dibagi dengan $\sqrt{d_{attn}}$, seperti pada \equ~\ref{equ:scaled-dot-product-attention}.

	\begin{align}
		\label{equ:scaled-dot-product-attention}
		f_{attn}(\mathbf{q}, \mathbf{k}) &= \frac{\mathbf{q} \mathbf{W}^q (\mathbf{k} \mathbf{W}^k)^{\top}}{\sqrt{d_{attn}}} \in \mathbb{R}, \\
		\text{dengan } &\mathbf{W}^q \in \mathbb{R}^{d_q \times d_{\text{attn}}}, \mathbf{W}^k \in \mathbb{R}^{d_k \times d_{\text{attn}}}.
	\end{align}



	pembagian dengan $\sqrt{d_{attn}}$ dilakukan menjaga variansi dari nilai atensi sehingga tetaplah sama dengan variansi dari vektor kueri $\mathbf{qW}^q$ dan vektor kunci $\mathbf{kW}^k$ (\f{pre-attention layer}). Seperti yang dapat dilihat pada \equ~\ref{equ:initialize-dot-product-attention} dan \equ~\ref{equ:variance-dot-product-attention}, hasil kali skalar antara vektor kueri $\mathbf{qW}^q$ dan vektor kunci $\mathbf{kW}^k$ memiliki faktor tambahan $\sigma^2 d_{attn}$, sehingga variansi dari nilai atensi $\mathbf{qW}^q (\mathbf{kW}^k)^{\top}$ adalah $\sigma^4 d_{attn}$. Pembagian dengan $\sqrt{d_{attn}}$ dilakukan untuk menghasilkan variansi yang sama dengan variansi dari vektor kueri $\mathbf{qW}^q$ dan vektor kunci $\mathbf{kW}^k$ seperti yang ditunjukkan pada \equ~\ref{equ:variance-scaled-dot-product-attention}. Faktor $\sigma^2$ tidaklah terlalu penting karena biasanya kita menjaga nilai dari $\sigma^2 \approx 1$, sehingga $\sigma^4 \approx 1$.

	\begin{align}
		\label{equ:initialize-dot-product-attention}
		\mathbf{qW}^q \sim \mathcal{N}(0, \sigma^2) \text{ dan } \mathbf{kW}^k \sim \mathcal{N}(0, \sigma^2). \\
		\label{equ:variance-dot-product-attention}
		\text{Var}(\mathbf{qW}^q (\mathbf{kW}^k)^{\top}) = \sum_{i=1}^{d_{attn}} \text{Var}\left((\mathbf{qW}^q)_i ((\mathbf{kW}^k)^{\top}_i\right) = \sigma^4 d_{attn} \\
		\label{equ:variance-scaled-dot-product-attention}
		\text{(scaled dot product attention) }\text{Var}\left(\frac{\mathbf{qW}^q (\mathbf{kW}^k)^{\top}}{\sqrt{d_{attn}}}\right) = \frac{\sigma^4 d_{attn}}{d_{attn}} = \sigma^4
	\end{align}

	Jika kita tidak menghilangkan faktor $d_{attn}$ , softmax dari nilai atensi akan jenuh ke 1 untuk satu elemen acak dan 0 untuk yang lainnya. Akibatnya, gradien pada fungsi softmax akan mendekati nol sehingga model tidak dapat belajar parameter dengan baik.

	\subsection{\f{Multi-Head Self-Attention}}
	
	\f{Multi-Head Attention} adalah arsiktetur \f{deep learning} yang melakukan mekanisme \f{attention} sebanyak
	
	\subsection{\f{Self-Attention}}
	
	\subsection{\f{Positional Encoding}}

	\subsection{\f{Position-wise Feed-Forward Network}}

	\subsection{Transformer Encoder}

\section{Bidirectional Encoder Representations from Transformers (BERT)}
	\subsection{Representasi Input}

	\subsection{Model Pralatih BERT}

		\subsubsection{\f{Masked Language Model}}

		\subsubsection{\f{Next Sentence Prediction}}

	\subsection{BERT untuk Bahasa Indonesia (IndoBERT)}

	\subsection{Penggunaan BERT untuk Pemeringkatan Teks}
		\subsubsection{$\text{BERT}_{\text{CAT}}$}

		\subsubsection{$\text{BERT}_{\text{DOT}}$}




