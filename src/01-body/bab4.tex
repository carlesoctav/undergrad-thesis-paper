\chapter{\babEmpat}
\label{bab:4}

Bab ini membahas mengenai proses fine tuning model Bidirectional Encoder Representations from Transformers (BERT) untuk mendapatkan model yang dapat digunakan untuk masalah pemeringkatan teks.
\sect\ref{sec:spesifikasi} menjelaskan mengenai spesifikasi perangkat keras dan perangkat lunak yang digunakan dalam penelitian. Selanjutnya, \sect~\ref{sec:simulasi} menjelaskan mengenai tahapan simulasi yang dilakukan dalam penelitian. Dataset latih (train) dan uji (validation) dijelaskan pada \sect~\ref{sec:dataset}. \sect~\ref{sec:finetuning} menjelaskan lebih detail mengenai arsitektur model BERT,fungsi loss, serta konfigurasi hyperparameter yang digunakan dalam proses fine tuning model BERT. \sect~\ref{sec:metriksbab4} menjelaskan kembali mengenai metriks evaluasi yang digunakan pada setiap dataset uji yang digunakan. Terakhir, \sect~\ref{sec:hasil} menjelaskan mengenai hasil fine tuning model BERT dan evaluasi dari model-model yang dihasilkan.


\section{Spesifikasi Mesin dan Perangkat Lunak}
\label{sec:spesifikasi}

\todo{
    banyak sih :'D
}

Proses fine tuning model BERT untuk pemeringkatan teks dilakukan menggunakan mesin dan perangkat lunak yang tertera pada  berikut.


\section{Tahapan Simulasi}
\label{sec:simulasi}

menunjukkan tahapan simulasi yang dilakukan dalam penelitian ini.




\section{Dataset Latih dan Uji}
\label{sec:dataset}

\subsection{Dataset Latih}
\label{sec:datasetlatih}

\subsubsection{Mmarco Indonesia Train Set}


\subsection{Dataset Uji}
\label{sec:datasetuji}

\subsubsection{Mmarco Indonesia DEV Set}

\subsubsection{Mrtydi Indonesia TEST Set}

\subsubsection{Miracl Indonesia TEST Set}

\section{Metriks Evaluasi}
\label{sec:metriksbab4}


\section{Fine Tuning BERT}
\label{sec:finetuning}

\subsection{$\text{IndoBERT}_{\text{CAT}}$}

\subsection{$\text{IndoBERT}_{\text{DOT}}$}

\subsection{$\text{IndoBERT}_{\text{DOT}}$+Hardnegs}

\subsection{$\text{IndoBERT}_{\text{DOTMargin}}$}

\subsection{$\text{IndoBERT}_{\text{KD}}$}


\section{Hasil Fine Tuning dan Evaluasi}
\label{sec:hasil}

\subsection{Evaluasi BM25}

\subsection{Evaluasi $\text{IndoBERT}_{\text{MEAN}}$}

\subsection{Evaluasi $\text{IndoBERT}_{\text{CAT}}$}

\subsection{Evaluasi $\text{IndoBERT}_{\text{DOT}}$}

\subsection{Evaluasi $\text{IndoBERT}_{\text{DOTHardnegs}}$}

\subsection{Evaluasi $\text{IndoBERT}_{\text{DOTMargin}}$}

\subsection{Evaluasi $\text{IndoBERT}_{\text{KD}}$}
